\documentclass[a4paper,french]{article}
\usepackage[utf8]{inputenc}
\usepackage[T1]{fontenc}
\usepackage{babel}
\usepackage[margin=2cm]{geometry}
\usepackage{listings}

\newcommand{\ttt}[1]{\texttt{#1}}

\title{Projet Programmation 2 -- Rapport de la partie 3\\Multijoueur et r\'eseau}
\author{\textsc{Daby-Seesaram} Arnaud, \textsc{Villani} Neven}
\date{}

\begin{document}
\maketitle

\section{Probl\`emes préliminaires rencontr\'es}

Bien que nous ayons beaucoup s\'epar\'e notre code en fichiers pour chaque aspect du jeu,
nous avons d\'ecouvert en d\'ebutant cette partie que certains aspects \'etaient trop
m\'elang\'es.

Par exemple notre classe \ttt{Pos} s'occupait \`a la fois d'aspects UI (r\'eactions aux
clics, affichage graphique) et logiques (ensemble d'organismes par case, items, combats).
Plus difficile encore, \ttt{BodyPart} s'occupait \`a la fois d'\'el\'ements relatifs au
jeu (items actifs, organismes vivants) et au joueur (progr\`es de l'objectif, position).\\

Il nous a donc fallu commencer par d\'ecoupler ces r\^oles:
\begin{itemize}
    \item \ttt{Room}, \ttt{BodyPart}, \ttt{Game}, \ttt{Pos}: logique du jeu
    \item \ttt{DisplayGrid}, \ttt{DisplayPos}: affichage graphique
    \item \ttt{LocalRoom}, \ttt{LocalPos}: dialogue entre \ttt{Room},\ttt{Pos} et \ttt{DisplayRoom},\ttt{DisplayPos}
\end{itemize}

\ttt{Local*} contiennent les informations absolument minimales pour l'affichage graphique, \ttt{Display*} en sont
juste des versions locales mises \`a jour r\'eguli\`erement.

Nous avons par ailleurs profit\'e de cette s\'eparation pour extraire une partie de
\ttt{BodyPart} dans \ttt{Game}, contenant tout ce qui est relatif \`a un seul joueur.
Nous avons fait attention \`a ce que toutes ces modifications rendent \ttt{BodyPart}
agnostique au nombre de \ttt{Game} qui existent, et donc en particulier qu'une fois la
partie r\'eseau mise en place il soit facile de rendre le jeu multijoueur.
\`A part une chute de performance, rien n'emp\^eche de jouer \`a 3 joueurs voire plus.\\

Cette s\'eparation a pris la majeure partie du temps allou\'e \`a la troisi\`eme partie,
nous aurions aim\'e avoir un peu plus de temps pour d'autres am\'eliorations mineures.



\section{R\'ealisation et ajouts}

C\^ot\'e client, un thread est responsable de dialoguer avec le serveur, parser les donn\'ees re\,cues, et
s\'erialiser les donn\'ees \`a envoyer, un autre met \`a jour l'affichage graphique et enregistre les
\'ev\`enements (clics, mouvements, commandes).\\

C\^ot\'e serveur il se passe le m\^eme processus de s\'erialisation et d\'es\'erialisation, et apr\`es chaque
\'etape du jeu la grille est lue et envoy\'ee \`a chaque client, puis une courte pause est faite jusqu'\`a-ce que
tous les clients aient confirm\'e la bonne r\'eception et \'eventuellement envoy\'e des commandes \`a ex\'ecuter.\\

L'affichage graphique distingue les virus du joueur de ceux des autres joueurs, mais ne distingue pas les 
adversaires les uns des autres dans le cas de trois joueurs ou plus.

Le syst\`eme de niveaux et d'objectifs a \'et\'e conserv\'e : chaque joueur doit compl\'eter les missions,
et une fois que l'un d'eux a atteint 100\% chaque joueur gagne autant de points que le pourcentage d'avancement
de sa mission. Une fois le dernier niveau termin\'e le joueur avec le plus de points gagne.

Heureusement les objectifs ainsi que la cr\'eation de niveaux ont \'et\'e tr\`es faciles \`a rendre multijoueurs.\\

Au bout d'un certain temps, si un joueur est inactif il est d\'eclar\'e forfaitaire : le jeu reprend sans lui.\\
Ses virus sont toujours g\'en\'er\'es, mais son curseur ne bouge plus.\\


\section{Installation}

La proc\'edure de lancement du jeu (et la structure du projet) ont \'et\'e un peu compliqu\'ees par l'ajout d'un serveur,
mais elles sont \'egalement des preuves que la s\'eparation serveur/client a \'et\'e bien r\'ealis\'ee.

\ttt{server/} contient le code qui ex\'ecute le jeu, \ttt{client/} contient la partie graphique.\\

L'exemple ci-dessous d\'ecrit comment lancer une partie \`a deux joueurs, il s'\'etend facilement \`a 1 ou 3 joueurs.\\

\`A ex\'ecuter une seule fois:
\begin{lstlisting}
$ cp -r player{-example,1}
$ cp -r player{-example,2}
$ echo "localhost 8888 20000" > player1/client.cfg
$ echo "localhost 8889 20000" > player2/client.cfg
$ echo "8888\n8889" > server/server.cfg
\end{lstlisting}~\\


\`A ex\'ecuter \`a chaque nouvelle partie, dans trois shells diff\'erents:
\begin{lstlisting}
$ cd server && sbt run
$ cd player1 && sbt run
$ cd player2 && sbt run
\end{lstlisting}~\\

Le dossier \ttt{src} de \ttt{player-example} est un symlink vers \ttt{client/src}, de m\^eme que nous avons choisi
que \ttt{client/**/communication.scala} soit un symlink vers \ttt{server/**/communication.scala} pour que la partie
dialogue entre le serveur et le client n'ait \`a \^etre modifi\'ee qu'\`a un seul endroit.\\
En particulier cela permet sans aucune duplication de code que \ttt{client} et \ttt{server} partagent la m\^eme classe
\ttt{Connection}, et que les s\'erialisations et d\'es\'erialisations de messages soient group\'ees ensemble.


\section{Limitations et am\'eliorations envisag\'ees}

\begin{itemize}
    \item nous avons envisag\'e de r\'etablir un syst\`eme de \ttt{play}/\ttt{pause} pour pouvoir interrompre le jeu,
        possiblement lui ajouter des restrictions pour emp\^echer un utilisateur de g\^ener les autres en mettant le
        jeu en pause \`a r\'ep\'etition
    \item les items qui sont envoy\'es entre le serveur et le client sont des cha\^ines de caract\`eres avec s\'eparateurs
        fix\'es comme \ttt{"///"} et \ttt{"|||"}. Cela n'est pas dangereux car l'utilisateur ne peut envoyer
        que des commandes qui sont valid\'ees par \ttt{Command} et les messages sont encapsul\'es dans un block \ttt{try},
        mais si un ajout d'une autre fonctionnalit\'e permettait au client d'envoyer au serveur autre chose que des commandes
        il faudrait ajouter d'autres m\'ecanismes de validation pour \'eviter des injections sous la forme de l'envoi
        d'une commande \ttt{"|||<commande malicieuse>"} qui serait d\'ecoup\'ee sur le \ttt{"|||"} et interpr\'et\'ee
        comme une commande vide provenant de l'utilisateur suivie d'un message provenant du client.
    \item certaines commandes pourraient \^etre d\'eplac\'ees c\^ot\'e client : il pourrait ne pas y avoir besoin de faire
        appel au serveur pour utiliser \ttt{help}
    \item les commandes \ttt{load}, \ttt{save} et \ttt{level} sont interdites si il y a deux joueurs ou plus, on pourrait les
        autoriser si tous les joueurs sont en faveur. Cela supposerait de mettre \`a jour le
        format de fichier pour accomoder plusieurs joueurs.
\end{itemize}



\end{document}

