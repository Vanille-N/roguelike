\documentclass[a4paper,french]{article}
\usepackage[utf8]{inputenc}
\usepackage[T1]{fontenc}
\usepackage{babel}
\usepackage[margin=2cm]{geometry}

\title{Projet Programmation 2 -- Rapport de la partie 2}
\author{\textsc{Daby-Seesaram} Arnaud, \textsc{Villani} Neven}
\date{}

\begin{document}
\maketitle

\section{Diversification}

\subsection{Objets}
Dans la première version du projet, nous disposions de deux types d'objets non
exclusifs : des objets mouvants, lesquels peuvent agir sans être ramassés et se
déplaçant de manière rectiligne et des objets statiques, lesquels doivent être
ramassés par des organismes pour être utilisés.

Dans cette seconde partie a été introduite une nouvelle famille d'objets
indépendante des deux premières : la famille des artefacts.
Un artefact peut apparaître lorsqu'un item meurt. Plus le temps passe dans la
partie, moins la probabilité de matérialisation d'artefact est grande.

Un artefact est un objet statique dont la fonction première est de modifier les
items environnant à l'aide des actions suivantes :
\begin{itemize}
    \item incrémenter par un le niveau d'un item
    \item décrémenter par un le niveau d'un item
    \item imposer un niveau à un item.
    \item multiplier par deux le niveau d'un item
    \item diviser par deux le niveau d'un item
    \item ne rien faire (artefact passif)
\end{itemize}

Ces actions sont effectuées sur tous les items, portés ou non se trouvant dans
un cercle autour de l'artefact.

Notons qu'il est ainsi possible d'obtenir des items de niveau 6 au maximum afin
d'éviter qu'ils n'agissent trop fortement sur le jeu, car un item possédant un
grand pouvoir demande de grands sacrifices pour qu'un porteur puisse l'utiliser
(si l'item agit sur la santé, l'organisme pourrait mourir instantanément pour
vouloir utiliser l'item).

\medskip

Il existe au sein des artefacts, pour tous les comportements ci-dessus cinq
familles: les artefacts meurtriers, qui tuent les organismes les approchant,
les artefacts forçant l'utilisation d'un item, quels que soient les risques
encourus par l'organisme, les artefacts proposant à chaque porteur d'item
d'utiliser ses items et les artefacts forçant tout porteur d'item à déposer ses
items, ce qui empêche dès lors le joueur de récolter ces items dans sont
inventaire.
\medskip

Les artefacts ne sont visibles que lorsqu'ils s'activent, pour en
faire une menace fantôme. Le joueur contrôlant un grand nombre de virus, et
pouvant observer les virus doit donc déduire de ce qu'il voit ou apparaissent
les artefacts et de quel genre d'artefact il s'agit pour déterminer sa
stratégie.

Les artefacts disparaissent au bout d'un certain temps déterminé par leur
niveau.


\subsection{Entit\'es}
Nous avons fait usage du m\'ecanisme de param\'etrage d'ennemis mis en place en partie 1 pour ajouter facilement trois nouveaux
types d'ennemis, dont deux fuyards et un aggressif, avec des caract\'eristiques vari\'ees (l'un est rapide, un autre est r\'esistant, le troisi\`eme
inflige beaucoup de d\'eg\^ats). Ils contribuent \`a rendre les niveaux de plus en plus difficiles.\\
Ils s'ajoutent aux trois types d'organismes (un fuyard, un aggressif, un passif) qui \'etaient d\'ej\`a pr\'esents dans la partie 1.\\

Il n'a pas \'et\'e ajout\'e de type de cellule compl\`etement diff\'erent (les trois nouveaux types h\'eritent du m\^eme type de base que
les cellules qui avaient \'et\'e introduites dans la partie 1), mais deux de nos cat\'egories d'items (items spaciaux qui se d\'eplacent et agissent sur
les organismes, artefacts qui agissent sur les organismes et autres items) ont des comportements qui en pratique
ont une complexit\'e proche de celle d'un organisme.

\subsection{Interactions}
Le joueur pouvait à l'issu de la partie 1 contrôler une sélection de virus, afin
de leur attribuer une autre cible que le pointeur. Il est à présent possible de
d'effectuer plusieurs sélections distinctes, et de leur donner des noms. Il
existe par défaut une sélection \texttt\_. Il est possible de redéfinir une
sélection en sélectionnant à l'écran des zones, de détruire des sélections,
d'afficher une sélection, de lister les sélections, de sélectionner tous les
organismes dans une sélection\footnote{Notons qu'un organisme peut se trouver
dans un nombre arbitraire de sélections.} et de récupérer tous les items des
virus d'une sélection pour le remettre à l'inventaire du joueur.\\

Additionnellement, il est donn\'e au joueur un cont\^ole plus fin sur le comportement des organismes
vis-\`a-vis des items avec l'enrichissement de la commande \texttt{behavior} par les sous-commandes
\texttt{give} et \texttt{keep} qui imposent aux virus de respectivement donner au joueur ou garder 
tous les items qu'ils ramassent.\\
Cela facilite l'usage des items pour une nouvelle fonctionnalit\'e qui leur a \'et\'e accord\'ee: le sacrifice.\\
Via la commande \texttt{sacrifice}, le joueur peut choisir d'abandonner tous les virus actuellement vivants et tous
les items qu'il poss\`ede dans son inventaire en \'echange de points de sacrifice. Ces points fournissent des am\'eliorations
permanentes aux stats des virus qui apparaissent, et ce \`a partir de la compl\'etion du niveau.

\section{Niveaux et objectifs}
Le joueur peut maintenant \'evoluer dans plusieurs niveaux, de mani\`ere plus ou moins s\'equentielle.\\
Le d\'eblocage du niveau \(n+1\) se fait par la compl\'etion de l'objectif du niveau \(n\), mais le joueur peut \'egalement
choisir de revenir en arri\`ere sur un niveau pr\'ec\'edent (plus facile) avec la commande \texttt{level}.\\
Revenir en arri\`ere n'est jamais n\'ecessaire pour le progr\`es du jeu, mais il peut \^etre plus facile de gagner des items
utilisables comme sacrifices dans des niveaux faciles, pour avoir des meilleures stats dans les niveaux plus difficiles.\\
En pratique il est tr\`es facile de compl\'eter tous les niveaux pour peu qu'on passe un peu de temps au niveau 1 pour
accumuler aux alentours de 20 points de bonus dans chaque stat, \`a l'inverse les niveaux \texttt{maze} et \texttt{brain} seront
tr\`es difficiles \`a vaincre sans des virus am\'elior\'es.\\

Les objectifs sont vari\'es: conqu\'erir une ou plusieurs positions (l'atteindre puis rester dessus avec au moins un certain nombre d'organismes),
tuer des cellules (quelconques ou d'une cat\'egorie particuli\`ere), ramasser des items d'un type particulier.
Ils sont de difficult\'e approximativement croissante.\\

\section{Syst\`eme de sauvegarde}
Une sauvegarde conserve:
\begin{itemize}
    \item le niveau maximal d\'ebloqu\'e
    \item le contenu de l'inventaire du joueur
    \item les stats actuelles des virus
\end{itemize}

Puisque le jeu s'organise en niveaux qui sont globalement courts et ind\'ependants, nous avons consid\'er\'e que recommencer au d\'ebut
du dernier niveau ne faisait pas perdre beaucoup de la progression du jeu.\\
Nous avons aussi estim\'e que permettre de reprendre le jeu dans un \'etat arbitraire du niveau serait trop difficile \`a mettre en place.


\section{Bilan sur les pr\'evisions de fin de premi\`ere partie}

\`A l'issue de la premi\`ere partie nous avions annonc\'e quelques plans de futurs ajouts. En voici le bila.

\begin{itemize}
    \item \textit{Plus de salles, niveaux compos\'es de plusieurs salles:} \textbf{Partiellement}\\
        Nous avons en effet ajout\'e de nombreuses salles, mais les niveaux ne pr\'esentent pas de m\'ecanisme de scrolling qui ferait passer d'une salle \`a une autre sans changement de niveau.

    \item \textit{Un m\'ecanisme (possiblement un item \`a ramasser) qui donne acc\`es au niveau suivant:} \textbf{D\'epass\'e}\\
        Non seulement il a \'et\'e impl\'ement\'e le fait que le rammassage d'un item particulier puisse faire passer au niveau suivant, nous avons m\^eme une bien plus grande diversit\'e d'objectifs.

    \item \textit{Une population de cellules (amicales et hostiles) plus diverse:} \textbf{Oui}\\
        Les cellules hostiles (et fuyardes) ont \'et\'e diversifi\'ees, les cellules amicales ont acquiv via le sacrifice des mani\`eres que leurs stats \'evoluent au cours du jeu, ce qui est une forme de diversification de leurs caract\'eristiques.

    \item \textit{Des items consommables qui fournissent des bonus temporaires ou permanents:} \textbf{Oui}\\
        Certains items sont consommables et donnent des bonus qui durent jusqu'\`a la mort de la cellule, le sacrifice donne des bonus encore plus permanents qui s'appliquent \`a des virus pas encore cr\'e\'es, les artefacts ont des effets qui s'apparentent \`a des bonus ou p\'enalit\'es temporaires aussi bien que permanents.

    \item \textit{Des items consommables qui fournissent des imunnit\'es \`a d'autres items ou organismes:} \textbf{Partiellement}\\
        Les artefacts ne sont pas consommables, mais ils peuvent agir sur les items de leur zone.
\end{itemize}
\end{document}

