\documentclass[a4paper,french]{article}
\usepackage[utf8]{inputenc}
\usepackage[T1]{fontenc}
\usepackage{babel}

\title{Fichier de mise en exergue du bon respect des consignes}
\author{}
\date{}

\begin{document}
\maketitle

\section{Première partie}
\subsection{Actions du joueur}
\begin{itemize}
\item
Personnage du jeu:
le personnage choisit est le curseur (case blanchie).

\item
Déplacements:
Lorsque les touches directionnelles ou hjkl sont pressées, le curseur se
déplace de manière adéquate sur le terrain.

\item
Ramasser des objets:
Cela s'effectue en volant les objets d'une sélection de virus. Le mécanisme
précis est détaillé dans l'aide.

\item
Actions réalisables par le joueur:
\begin{enumerate}
	\item
	Modification manuelle des statistiques d'un organisme (permet de (dé)faire
	un super virus ou un super ennemi).

	\item
	Donner un item à un organisme précis. Ce dernier pourra l'utiliser de
	manière automatique lorsqu'il le décidera.

	\item
	Forcer l'utilisation d'un item par un organisme.

	\item
	Décider de la stratégie adoptée par une sous-partie des virus (quelle est la
	case vers laquelle se dirigée).
\end{enumerate}
\end{itemize}





\subsection{Les objets et les monstres}
\begin{itemize}
\item

\end{itemize}






\subsection{Discrétisation}
\subsection{Graphique}



\section{Seconde partie}

\end{document}

